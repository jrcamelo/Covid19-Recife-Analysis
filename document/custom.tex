
% ----------------------------------------------------------
% DADOS DO TRABALHO - CAPA e FOLHA DE ROSTO
% Configure os dados do trabalho aqui
% ----------------------------------------------------------


\titulo{\textbf{Análise de Dados Públicos da COVID-19 em Recife \\ Utilizando Aprendizagem de Máquina}}
\autor{JOÃO RAFAEL SANTOS CAMELO}
\local{Recife}
\data{\Year}
\orientador{\textbf{Orientador}: Fernando Maciano de Paula Neto}

\instituicao{UNIVERSIDADE FEDERAL DE PERNAMBUCO \\ CENTRO DE INFORMÁTICA \\ GRADUAÇÃO EM SISTEMAS DE INFORMAÇÃO}
\departamento{Centro de Informática}
\programa{Graduação em Sistemas de Informação}
\emailprograma{jrsc2@cin.ufpe.br}
\siteprograma{http://cin.ufpe.br/\textasciitilde posgraduacao}

\tipotrabalho{Trabalho de Graduação}
% O preambulo deve conter o tipo do trabalho, o objetivo, 
% o nome da instituição e a área de concentração 
%\preambulo{Trabalho apresentado ao Programa de Pós-graduação em Ciência da Computação do Centro de Informática da Universidade Federal de Pernambuco, como requisito parcial para obtenção do grau de Mestre Profissional em Ciência da Computação.}

%\preambuloatadefesa{Dissertação apresentada ao Programa de Pós-Graduação Profissional em Ciência da Computação da Universidade Federal de Pernambuco, como requisito parcial para a obtenção do título de Mestre Profissional em 04 de setembro de 2020.}

\preambulo{Trabalho apresentado ao Programa de Graduação em Sistemas de Informação do Centro de Informática da Universidade Federal de Pernambuco como requisito parcial para obtenção do grau de Bacharel em Sistemas de Informação}

\preambuloatadefesa{Monografia apresentada ao programa de Graduação em Sistemas de Informação do Centro de Informática da Universidade Federal de Pernambuco, sob o título Análise de Dados Públicos da COVID-19 em Recife Utilizando Aprendizagem de Máquina, orientada pelo Prof. Dr. Fernando Maciano de Paula Neto.}




\input{userlists}



