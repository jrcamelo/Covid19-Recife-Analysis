\chapter{Contexto}
\label{chap:contexto}

Este capítulo introduz temas que servem como base para a análise e o desenvolvimento presentes nesse estudo.

A seção \ref{sec:ia-na-saude} reflete sobre os avanços, tanto atuais quanto possíveis, providos pela aplicação de inteligência artificial na área de saúde. Em seguida, a seção \ref{sec:algoritmos} explica o funcionamento teórico dos modelos de aprendizado de máquina utilizados no desenvolvimento deste trabalho. Por fim, a seção \ref{sec:trabalhos} contém alguns trabalhos relacionados ao presente estudo, servindo como base acadêmica.


\section{Aprendizado de Máquina na Saúde}
\label{sec:ia-na-saude}

Aprendizado de máquina é um ramo da inteligência artificial e ciência da computação que foca no uso de dados e algoritmos para imitar como os humanos aprendem, melhorando gradativamente sua precisão \cite{ibm}. Algoritmos de aprendizado de máquina são então treinados para reconhecer padrões em dados ou mídias. A partir disso, é possível obter resultados como classificação de dados que não conheciam anteriormente, ou produção de conteúdo se baseando nos padrões percebidos.

Duas áreas da medicina que se beneficiam do aprendizado de máquina são as de diagnóstico e prognóstico de doenças. Diagnóstico consiste em avaliar o estado atual do paciente, tendo visto avanços, por exemplo, utilizar fotos para identificar câncer de pele. Prognóstico consistem em prever a evolução da doença no paciente, como utilizar a coleta de dados de tecnologias vestíveis em pacientes com diabetes para ajudar no tratamento \cite{ml-medicina}.

Uma conquista recente para a medicina veio da rede de inteligência artificial da \textit{Google}, \textit{DeepMind}. Seu algoritmo, \textit{AlphaFold2}, conseguiu prever precisamente a estrutura tridimensional de proteínas a partir da sequência de aminoácidos, um problema enfrentado na área há décadas \cite{deepmind}. Este avanço possibilita um melhor entendimento de como proteínas se comportam e pode trazer uma mudança de paradigma na fabricação de remédios e na compreensão do funcionamento de células \cite{deepmind2}.

O uso de aprendizado de máquina depende da análise de grandes quantidades de dados, mas a emergência de tecnologias vestíveis e sistemas mais distribuídos trazem bons prospectos para seu uso na área de saúde \cite{wearables}.


% \input{chapters/contexto/sections/trabalhos}

\section{Comparação entre Algoritmos}
\label{sec:comparacao-algoritmos}

Após uma certa quantidade de iterações do processo de otimização de parâmetros como especificado na subseção \ref{subsec:otimizacao}, os modelos gerados pelos algoritmos de classificação \textit{k-Nearest Neighbors} (kNN), \textit{Decision Tree} (DT), \textit{Random Forest} (RF), \textit{Gradient Boosting} (GB), \textit{Light Gradient Boosting} (LGB) e \textit{XGBoost} (XGB) %e \textit{Neural Network} (NN)% 
na etapa de treinamento foram aplicados no conjunto de dados de teste.

Cada algoritmo foi executado 20 vezes, aleatorizando o conjunto de dados em cada uma delas, coletando então suas métricas de acurácia, precisão, sensibilidade, \textit{F1-Score} e \textit{AUC-ROC} para cada execução. 
Como discutido na subseção \ref{subsec:calculo-metricas}, as métricas macro representam a média dos resultados de cada classe, que neste conjunto de dados são casos leves e casos graves. 
A média e desvio padrão de cada métrica foi calculada, e as métricas consideradas relevantes para comparação se encontram na tabela \ref{tab:comparacao-algoritmos-normal}, onde o maior valor para cada uma está representado em negrito.

\begin{table}[H]
  \footnotesize
  \centering
  \begin{tabular}{l|c|c|c|c|c|}
  \cline{2-6}
  \textbf{}                          & \textbf{Acurácia}      & \textbf{Precisão Macro} & \textbf{Sensib. Macro} & \textbf{F1-Score Macro} & \textbf{AUC-ROC}       \\ \hline
  \multicolumn{1}{|l|}{\textbf{kNN}} & 0,9790±0,0001          & 0,9475±0,0007           & 0,8228±0,0020          & 0,8739±0,0011           & 0,8228±0,0020          \\ \hline
  \multicolumn{1}{|l|}{\textbf{DT}}  & 0,9767±0,0015          & 0,9558±0,0133           & 0,7913±0,0094          & 0,8534±0,0101           & 0,7913±0,0094          \\ \hline
  \multicolumn{1}{|l|}{\textbf{RF}}  & 0,9828±0,0001          & \textbf{0,9795±0,0006}  & 0,8395±0,0012          & 0,8962±0,0009           & 0,8395±0,0012          \\ \hline
  \multicolumn{1}{|l|}{\textbf{GB}}  & 0,9835±0,0002          & 0,9727±0,0022           & 0,8513±0,0015          & 0,9020±0,0012           & 0,8513±0,0015          \\ \hline
  \multicolumn{1}{|l|}{\textbf{LGB}} & 0,9835±0,0002          & 0,9748±0,0016           & 0,8498±0,0015          & 0,9018±0,0001           & 0,8498±0,0015          \\ \hline
  \multicolumn{1}{|l|}{\textbf{XGB}} & \textbf{0,9838±0,0003} & 0,9707±0,0010           & \textbf{0,8571±0,0031} & \textbf{0,9052±0,0022}  & \textbf{0,8571±0,0031} \\ \hline 
\end{tabular}
\caption{Médias de métricas de algoritmos de classificação na etapa de teste}
\label{tab:comparacao-algoritmos-normal}
\end{table}

Levando em consideração a métrica \textit{AUC-ROC} como métrica de avaliação, é possível observar que, enquanto todos os algoritmos alcançaram um valor próximo ou acima de 80\%, o algoritmo \textit{XGBoost} obteve o melhor desempenho, bem próximo dos outros algoritmos de \textit{Gradient Boosting}. O \textit{F1-Score Macro} também é um indicador de desempenho satisfatório, e todos os algoritmos alcançaram valores acima de 85\%, com os algoritmos baseados em \textit{Gradient Boosting} alcançando valores acima de 90\%. 

Embora a Precisão Macro tenha alcançado valores altos, até acima de 97\%, a Sensibilidade Macro ficou sempre abaixo dos 90\%, devido a uma quantidade considerável de falsos negativos, casos graves classificados como leves. É possível observar este resultado na tabela \ref{tab:matriz-confusao-xgboost}, que mostra a matriz de confusão do \textit{XGBoost} com melhor \textit{F1-Score} das 20 execuções, 90,97\%.

\begin{table}[H]
  \footnotesize
  \centering
  \centering
  \begin{tabular}{l|c|c|}
    \cline{2-3}
    \textbf{}                         & \multicolumn{1}{l|}{\textbf{Predição: LEVE}} & \multicolumn{1}{l|}{\textbf{Predição: GRAVE}} \\ \hline
    \multicolumn{1}{|l|}{\textbf{Real: LEVE}}  & 114511                                       & 205 (0,18\%)                                           \\ \hline
    \multicolumn{1}{|l|}{\textbf{Real: GRAVE}} & 1664 (27,07\%)                                         & 4483                                          \\ \hline   
  \end{tabular}
  \caption{Matriz de confusão de uma execução do \textit{XGBoost}}
  \label{tab:matriz-confusao-xgboost}
\end{table}

Tomando a tabela \ref{tab:matriz-confusao-xgboost} como exemplo, somente 0,18\% de casos leves foram preditos como graves, um número ínfimo de falsos negativos. Isso significa que existe uma confiabilidade satisfatória nos casos preditos como graves. 

Contudo, 27\% dos casos graves foram erroneamente preditos como leves, significando que cerca de 1/4 dos casos graves não foram percebidos pelo algoritmo. Este fenômeno se mostrou presente em todos os algoritmos analisados, sendo a causa da diminuição da Sensibilidade Macro observada na tabela \ref{tab:comparacao-algoritmos-normal}.

Este problema ocorre devido ao desbalanceamento do conjunto de dados, onde existem 20x mais casos leves que graves. Portanto, durante a etapa de otimização de parâmetros descrita na subseção \ref{subsec:otimizacao}, técnicas de balanceamento de dados foram utilizadas. O processo de balanceamento consiste em igualar a quantidade de classes no conjunto de dados, excluindo registros aleatórios da classe mais comum. Os resultados se encontram na tabela \ref{tab:comparacao-algoritmos-undersample}.

\begin{table}[H]
  \footnotesize
  \centering
  \begin{tabular}{l|c|c|c|c|c|}
  \cline{2-6}
  \textbf{}                          & \textbf{Acurácia}      & \textbf{Precisão Macro} & \textbf{Sensib. Macro} & \textbf{F1-Score Macro} & \textbf{AUC-ROC}       \\ \hline
  \multicolumn{1}{|l|}{\textbf{kNN}} & 0,9047±0,0027          & 0,9050±0,0026           & 0,9047±0,0027          & 0,9047±0,0027           & 0,9047±0,0027          \\ \hline
  \multicolumn{1}{|l|}{\textbf{DT}}  & 0,8896±0,0102          & 0,8905±0,0096           & 0,8896±0,0102          & 0,8895±0,0103           & 0,8896±0,0102          \\ \hline
  \multicolumn{1}{|l|}{\textbf{RF}}  & 0,9150±0,0019          & 0,9158±0,0020           & 0,9150±0,0019          & 0,9150±0,0019           & 0,9150±0,0019          \\ \hline
  \multicolumn{1}{|l|}{\textbf{GB}}  & 0,9234±0,0022          & 0,9238±0,0022           & 0,9234±0,0022          & 0,9234±0,0022           & 0,9234±0,0022          \\ \hline
  \multicolumn{1}{|l|}{\textbf{LGB}} & 0,9233±0,0009          & 0,9239±0,0009           & 0,9233±0,0009          & 0,9233±0,0009           & 0,9233±0,0009          \\ \hline
  \multicolumn{1}{|l|}{\textbf{XGB}} & \textbf{0,9243±0,0016} & \textbf{0,9249±0,0013}  & \textbf{0,9243±0,0016} & \textbf{0,9242±0,0016}  & \textbf{0,9243±0,0016} \\ \hline
\end{tabular}
\caption{Médias de métricas de algoritmos de classificação na etapa de testes com classes balanceadas}
\label{tab:comparacao-algoritmos-undersample}
\end{table}

A Sensibilidade Macro teve um aumento considerável, chegando acima dos 90\% na maioria dos algoritmos, enquanto a Precisão Macro sofreu uma queda. O \textit{F1-Score} consequentemente teve um aumento, por ser a média harmônica dessas medidas. A acurácia caiu significativamente, não mais inflada pelos casos leves, enquanto o valor \textit{AUC-ROC} cresceu proporcionalmente. Demonstram-se essas diferenças mais detalhadamente na tabela \ref{tab:matriz-confusao-xgboost-undersample}, uma matriz de confusão do XGBoost com \textit{F1-Score} de 92,67\%.

\begin{table}[H]
  \footnotesize
  \centering
  \centering
  \begin{tabular}{l|c|c|}
    \cline{2-3}
    \textbf{}                         & \multicolumn{1}{l|}{\textbf{Predição: LEVE}} & \multicolumn{1}{l|}{\textbf{Predição: GRAVE}} \\ \hline
    \multicolumn{1}{|l|}{\textbf{Real: LEVE}}  & 5770                                       & 377 (6,13\%)                                           \\ \hline
    \multicolumn{1}{|l|}{\textbf{Real: GRAVE}} & 524 (8,52\%)                                         & 5623                                          \\ \hline   
  \end{tabular}
  \caption{Matriz de confusão de uma execução do \textit{XGBoost} com classes balanceadas}
  \label{tab:matriz-confusao-xgboost-undersample}
\end{table}

Neste cenário, a diferença entre a proporção de falsos positivos e falsos negativos se equilibra, ainda que se mantenha maior nos casos graves. Existe uma chance de que casos leves sejam preditos como graves, mas a chance de que casos graves sejam preditos como leves é muito menor que anteriormente. Considerando a natureza médica do problema, uma menor taxa de casos graves perdidos pode ser considerada uma melhoria significativa, mesmo que exista um aumento na quantidade de alarmes falsos para casos leves \cite{medical-ai-measure}. Portanto, assimilando este julgamento ao aumento do valor \textit{AUC-ROC}, escolhido como métrica de desempenho, balancear o conjunto de dados por \textit{RandomUnderSampler} se evidencia como uma melhoria.


\subsection{Cálculo de Métricas de Desempenho}
\label{subsec:calculo-metricas}

A etapa de cálculo de métricas de desempenho recebe a lista de previsões dos algoritmos e os valores reais do conjunto de dados, que são usados para gerar diversas métricas. Cada uma das métricas é discutida na seção \ref{sec:metricas}

De modo a acomodar as diferenças entre cada execução, as etapas de treinamento e teste são executadas 20 vezes para cada coleta de métricas, com conjuntos de treinamento e teste aleatórios em cada execução. O máximo, a média e o desvio padrão das métricas são calculados para cada algoritmo, e servem como referência na otimização e resultados do projeto. Este método é conhecido como \textit{k-fold cross-validation}, e permite avaliar os algoritmos de classificação de maneira mais consistente \cite{kfold}.







\section{Trabalhos Relacionados}
\label{sec:trabalhos}

Considerando a relevância de aplicar inteligência artificial na área de saúde e a urgência em escala global da pandemia de COVID-19, uma miríade de estudos sobre o tema foram publicados, tanto internacionalmente quanto no Brasil. Alguns trabalhos relacionados ao presente estudo são descritos a seguir. O valor \textit{AUC-ROC}, discutido na subseção \ref{subsec:calculo-metricas}, será utilizado como principal métrica de desempenho entre trabalhos.

\subsection{Trabalhos Internacionais}
\label{subsec:trabalhos-internacionais}

O trabalho \cite{yuanfang} analisou os dados de 214 pacientes de Wuhan, China, com informações sobre suas comorbidades, sintomas e resultados de testes de laboratório, a fim de prever a severidade de casos, de modo similar ao presente trabalho. Teve-se como objetivo manter a interpretabilidade da previsão, portanto, mesmo utilizando outros algoritmos como kNN, redes neurais e \textit{naive Bayes}, decidiu-se focar em \textit{Random Forest}, obtendo um \textit{AUC-ROC} de 99\% com os dados laboratoriais, e de 90\% com as comorbidades e sintomas. 

Utilizando \textit{Gradient Boosting}, o trabalho \cite{yazeed} analisou dados de 99.232 registros de indivíduos testados por COVID-19 em Israel, onde 8393 deles foram confirmados como portadores da doença. O objetivo em questão foi identificar portadores da doença por meio de registros de grupos de idade, sexo, 5 sintomas iniciais e se houve contato com alguém infectado, para que fosse possível priorizar recursos de teste limitados. Foi alcançado um \textit{AUC-ROC} de 90\% utilizando \textit{Gradient Boosting}. Como esperado pelos autores, o contato com alguém infectado foi o fator mais importante. Foi observado também que o Ministério da Saúde de Israel não registrou dados sintomáticos suficientes.

Conduzindo uma meta-análise acadêmica, o trabalho \cite{metaanalise} buscou determinar o grau em que comorbidades associaram-se com casos graves e óbitos a fim de auxiliar em medidas de tratamento, planejamento e provisionamento. Com um total de 26 estudos analisados e 13.400 amostras, identificaram que doenças pulmonares obstrutivas crônicas, cerebrovasculares, cardiovasculares, diabetes, câncer e hipertensão arterial foram as comorbidades mais significantes para casos graves de COVID-19. Também perceberam que nos estudos analisados, mesmo idade e sexo sendo fortes preditores de mortalidade, em termos de relação entre sintoma e comorbidades, hipertensão e diabetes se relacionaram a pneumonia, enquanto hipertensão se relacionou à síndrome da insuficiência respiratória aguda.

Sendo um dos trabalhos mais similares ao trabalho atual, a pesquisa \cite{mdmartuza} propôs analisar características dos pacientes, sintomas, diagnósticos e evolução da doença para descobrir os melhores preditores de diagnóstico precoce, possibilitando decisões rápidas nas necessidades de tratamento e isolamento. Como conjunto de dados, utilizou-se dados abertos de 6.512 pacientes de províncias da China. Foi alcançado uma \textit{AUC-ROC} de 89\% no algoritmo \textit{XGBoost}. Além deste, também foram utilizados \textit{Gradient Boosting}, \textit{Support Vector Machine}, \textit{Decision Tree} e \textit{Random Forest}, porém com desempenho inferior. Houve um esforço na filtragem dos dados em relação à idade, analisando o desempenho dos algoritmos em grupos de idade diferentes, com acurácia variável, mas conseguiu-se identificar as características mais relevantes para cada grupo.

Em comparação com estes trabalhos internacionais, o presente trabalho analisa casos confirmados, tanto prevendo a severidade quanto o óbito. Não serão utilizados dados laboratoriais, que precisam de mais recursos e tempo para obter. Ainda assim, o conjunto de dados disponibilizado pela Prefeitura do Recife, além de muito mais vasto, possui uma ampla variedade de sintomas e comorbidades, o que pode se mostrar um ponto positivo.


\subsection{Trabalhos no Brasil}
\label{subsec:trabalhos-brasileiros}

O trabalho \cite{igor} analisou 217.580 pacientes de Alagoas (AL), Espírito Santo (ES) e Santa Catarina (SC), de modo a prever casos onde o paciente precisaria de hospitalização. Os dados incluíram idade, sexo, raça, sintomas, comorbidades e a evolução do caso. Foram utilizados os algoritmos de \textit{Decision Tree}, redes neurais e \textit{Support Vector Machine}, alcançando \textit{AUC-ROC} médios de 87\%, 90\% e 91\% para AL, ES e SC. Um viés identificado pelos autores foi o de que estados com melhor infraestrutura geram dados mais confiáveis, refletido no desempenho maior do algoritmo conforme o Índice de Desenvolvimento Humano (IDH) dos estados. Também foi teorizado que a quantidade de leitos disponíveis tenha sido um fator de barulho relevante.

O trabalho \cite{betech} envolveu todas as regiões do Brasil, com 113.214 pacientes, onde 50.387 resultaram em óbito, teve como objetivo prever a mortalidade dos casos de COVID-19 no Brasil. Os dados foram similares aos do trabalho anterior \cite{igor}, tendo também informações de tratamento. Por meio de \textit{Support Vector Machine}, Regressão Logística, \textit{Gradient Boosted Decision Trees} e \textit{Random Forest}, foi possível obter \textit{AUC-ROC}s na predição da mortalidade e necessidade de hospitalização de 79\% e 69\% respectivamente. A região do hospital foi observada como um fator relevante. A região Nordeste teve a maior razão de probabilidade de mortalidade (2,185) entre todos os fatores, sendo mais que o dobro da região Sudeste (1,030).

Neste contexto, o presente trabalho analisará dados somente de Recife, na região Nordeste, deixando de lado diferenças de desenvolvimento. Serão analisados ambos os cenários de severidade e de óbito, como nestes estudos discutidos. Um fator não estudado anteriormente, tanto no Brasil quanto em outros países, é a vacinação em progresso, que receberá foco neste trabalho.