\chapter{Conclusão}
\label{chap:conclusao}

O objetivo principal deste trabalho foi avaliar a eficácia de algoritmos de classificação por aprendizagem de máquina na previsão de casos graves de COVID-19 e possível óbito, utilizando conjuntos de dados disponibilizados pela Prefeitura do Recife. Além disso, a evolução da doença na cidade e os efeitos da vacinação em progresso foram investigados por meio dos modelos gerados. O estudo foi motivado pelo cenário de pandemia e oportunidade de aproveitamento da abundância de dados coletados para estudar possíveis peculiaridades da doença em Recife e os impactos da vacinação.

Os algoritmos foram avaliados no conjunto de dados completo, balanceado por classes e filtrado com certos critérios como progresso de vacinação ou omissão de dados sintomáticos. As médias e desvios das métricas coletadas foram apresentadas em forma de tabela, utilizando \textit{AUC-ROC} como métrica de desempenho, auxiliada por \textit{F1-Score}. Os resultados mostram que todos os algoritmos de classificação experimentados são capazes de classificar os casos da doença e possível óbito. Foram alcançados médias de 92\% e 95\% de acurácia, \textit{F1-Score} e \textit{AUC-ROC} na predição de casos graves e óbitos respectivamente, um desempenho superior ao esperado quando inicialmente comparado com os trabalhos relacionados na seção \ref{sec:trabalhos}, que alcançaram entre 79\% e 95\% acurácia com outros conjuntos de dados. Filtrando por casos onde a vacinação estava mais avançada, a predição de óbitos obteve métricas ainda melhores, alcançando 98\% e 97\% nas médias de acurácia, \textit{F1-Score} e \textit{AUC-ROC} na predição de severidade e óbitos respectivamente. Métodos baseados em \textit{Gradient Boosting}, em especial o \textit{XGBoost}, mostraram um melhor desempenho entre os algoritmos de classificação avaliados.

Os algoritmos escolhidos foram interpretáveis, de modo que a investigação do funcionamento dos modelos gerados fosse possível. Gráficos \textit{SHAP} foram utilizados para identificar fatores de risco como idade avançada, doenças cardiovasculares e estilos de vida não saudáveis como tabagismo e alcoolismo. O progresso da vacinação se mostrou relevante na diminuição da severidade dos casos, como também na redução da probabilidade de óbito, interagindo com fatores de risco de maneira generalizada, em especial foi percebido uma diminuição dos casos graves em idades mais baixas.

Portanto, se conclui que é possível predizer com satisfatório grau de precisão a evolução de casos de COVID-19 em Recife utilizando dados públicos, por meio de algoritmos de classificação por aprendizagem de máquina. Fatores de risco e efeitos da vacinação podem ser identificados com auxílio dos modelos treinados.

\section{Trabalhos Futuros}
\label{sec:trabalhosfuturos}

Com o propósito de posteriormente estender o escopo do estudo, podem ser utilizados outros algoritmos de classificação como Redes Neurais Artificiais e \textit{Support Vector Machine}, vistos em estudos relacionados. A fim de expandir a análise dos conjuntos de dados disponíveis e melhorar o discernimento de fatores de risco e detecção de impacto de certos fatores como a vacinação, podem ser realizados experimentos de regressão ou aplicação de \textit{Deep Learning}. Pode também ser realizada melhor aplicação de técnicas estatísticas na avaliação dos resultados, por exemplo, testes de hipótese e análises de variância. Ajustes nos parâmetros de configuração dos algoritmos e pré-processamento dos conjuntos de dados podem também ser realizados para melhorar a precisão dos resultados obtidos, sendo possível também investigar a continuação da vacinação na cidade do Recife nos dados futuros, com uma maior porcentagem da população vacinada e informações de doses de reforço da vacina. 
