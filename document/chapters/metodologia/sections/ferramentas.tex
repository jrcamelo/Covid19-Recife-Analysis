\section{Ferramentas Utilizadas}
\label{sec:ferramentas}

O desenvolvimento e execução do projeto foi realizado por meio da linguagem \textit{Python}, na versão 3.8.5 \cite{python}. A motivação da escolha foi devido à sua relevante presença acadêmica na área de inteligência artificial. Além disso, há uma ampla disponibilidade de bibliotecas no tema, das quais as seguintes foram usadas:

\begin{itemize}
  \item \textit{NumPy}: \cite{numpy} - Ferramentas de manipulação de matrizes e vetores;
  \item \textit{Pandas}: \cite{pandas} - Ferramentas de manipulação de dados;
  \item \textit{ScikitLearn}: \cite{sklearn} - Algoritmos de aprendizado de máquina;
%  \item \textit{Keras}: \cite{keras} - Algoritmos de redes neurais;
  \item \textit{SHAP}: \cite{shap} - Ferramentas de análise de impacto e geração de gráficos;
  \item \textit{Seaborn}: \cite{seaborn} - Geração de gráficos de correlação;
  \item \textit{MatPlotLib}: \cite{matplotlib} - Geração de gráficos diversos;
\end{itemize}