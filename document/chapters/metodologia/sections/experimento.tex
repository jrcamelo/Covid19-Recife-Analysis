\section{Experimento}
\label{sec:experimento}

Após a obtenção e pré-processamento dos dados, foi realizado o experimento, dividido em seis etapas interligadas, discutidas detalhadamente a seguir.

O conjunto de dados é dividido em treinamento e teste de forma semi-aleatória, onde a frequência de cada classe é respeitada, separando-as proporcionalmente de acordo com a \textbf{Severidade}. Para isso são usados algoritmos de divisão de conjunto de dados do \textit{ScikitLearn} com estratégias de estratificação \cite{sklearn}. O resultado deste processo são dois conjuntos de dados separados, com 80\% dos dados no conjunto de treinamento e os 20\% restantes no conjunto de teste.

\subsection{Etapa de Treinamento}
\label{subsec:etapa-treinamento}

Na etapa de treinamento, os algoritmos de classificação são treinados utilizando o conjunto de treinamento e parâmetros de configuração. Estes parâmetros devem ser ajustados apropriadamente na etapa de otimização, explicada na subseção \ref{subsec:otimizacao}, sendo repetidas estas etapas e as seguintes conforme necessário. O resultado desta etapa são modelos especializados no problema de classificação apresentado pelo conjunto de dados.



\subsection{Etapa de Teste}
\label{subsec:etapa-teste}

A etapa de teste é onde os modelos gerados pelos algoritmos utilizam o conjunto de testes para tentar prever corretamente a classificação de cada caso. O conjunto de estes são os casos restantes após a separação do conjunto de treinamento. A entrada desta etapa é o conjunto de testes e o modelo gerado na etapa de treinamento, enquanto sua saída é uma lista de previsões, corretas ou não.

\subsection{Cálculo de Métricas de Desempenho}
\label{subsec:calculo-metricas}

A etapa de cálculo de métricas de desempenho recebe a lista de previsões dos algoritmos e os valores reais do conjunto de dados, que são usados para gerar diversas métricas. Cada uma das métricas é discutida na seção \ref{sec:metricas}

De modo a acomodar as diferenças entre cada execução, as etapas de treinamento e teste são executadas 20 vezes para cada coleta de métricas, com conjuntos de treinamento e teste aleatórios em cada execução. O máximo, a média e o desvio padrão das métricas são calculados para cada algoritmo, e servem como referência na otimização e resultados do projeto. Este método é conhecido como \textit{k-fold cross-validation}, e permite avaliar os algoritmos de classificação de maneira mais consistente \cite{kfold}.






\subsection{Otimização de Parâmetros}
\label{subsec:otimizacao}

A etapa de otimização de parâmetros consiste em ajustar os parâmetros de configuração dos algoritmos de classificação para que sejam o mais adequados o possível aos dados de treinamento. Para avaliar isso, são usadas as métricas obtidas na etapa anterior. De acordo com a necessidade, as etapas anteriores são executadas múltiplas vezes de modo a obter o melhor resultado.

Para facilitar esse processo, foi utilizada a técnica de otimização de parâmetros (\textit{grid search}) do \textit{ScikitLearn} \cite{sklearn}. O \textit{grid search} tenta encontrar os melhores parâmetros de forma automatizada, executando uma busca exaustiva com diversos parâmetros de cada algoritmo \cite{gridsearch}.

Notou-se que o conjunto de dados estava desbalanceado, ou seja, a quantidade de casos leves era 20 vezes maior que a quantidade de casos graves. Portanto, se viu necessário aplicar a técnica de balanceamento de conjunto de dados \textit{RandomUnderSampler}. Os registros da classe mais comum são removidos aleatoriamente para balancear as classes. Então a quantidade casos leves e torna igual à de casos graves no conjunto de dados balanceado.

\subsection{Geração de Gráficos}
\label{subsec:geracao-graficos}

Tendo um modelo do algoritmo de classificação satisfatório, foram gerados então diversos gráficos interpretáveis para a análise dos dados. Em especial os gráficos SHAP para entender a interpretação do modelo sobre o conjunto de dados na importância de cada variável \cite{shap}. 

\begin{figure}[ht!]
  \centering
  \fcolorbox{white}{white}{\includegraphics[width=0.5\textwidth]{chapters/resultados/images/xgboost_normal_dot.png}}
  \caption{\textmd{Gráfico SHAP de interpretação de relevância de atributos.}}
  \legend{Fonte: \textit{SHAP documentation}}
  \label{fig:shap-exemplo}
\end{figure}

Gráficos SHAP de pontos como na figura \ref{fig:shap-exemplo} de exemplo combinam gráficos de dispersão com estimativas de densidade. A cor dos pontos significa o valor do atributo, onde pontos azuis significam que o valor do atributo é baixo ou falso, enquanto pontos vermelhos significam que o valor é alto ou verdadeiro. A posição horizontal dos pontos representa o impacto do atributo no resultado da classificação, onde pontos na esquerda significam que aquele valor é importante na classificação negativa, e pontos na direita na classificação positiva.


Também é possível gerar gráficos de análise do conjunto de dados, como gráficos de Mapas de Calor de Correlação \cite{correlation} e Plotagens de Densidade das Colunas \cite{density}, que também podem servir como fonte de informação para a análise dos dados.

\subsection{Subconjuntos}
\label{subsec:subconjuntos}

Foi observado que o conjunto de dados permite diferentes cenários por meio de subconjuntos, podendo oferecer \textit{insights} sobre os resultados, fatores de risco e impactos das variáveis. Portanto, as seguintes análises são propostas.

\subsubsection{Omitir dados sintomáticos}
\label{subsubsec:omitindosintomas}

O conjunto de dados utilizado no experimento possui 20 colunas de dados sintomáticos apresentados pelos pacientes após confirmado o diagnóstico de COVID-19. Ao omitir essas colunas, é possível observar como os algoritmos predizem os casos leves e graves se baseando somente nos dados demográficos, doenças preexistentes e progresso da vacinação. Idealmente, será possível perceber fatores de risco demográficos e clínicos antes mesmo do paciente contrair COVID-19, permitindo uma melhor priorização de tratamento.

\subsubsection{Possibilidade de óbito}
\label{subsubsec:analisandoobito}

Cerca de 1/4 dos casos graves resultaram em óbito do paciente, como demonstrado na coluna \textbf{Data de Óbito}. Por meio desta, é possível comparar casos onde não houve óbito com casos que resultaram em óbito, em um conjunto de dados binário. Sendo assim, é possível identificar especificamente fatores de risco que podem ocasionar no óbito do paciente.

\subsubsection{Progresso da vacinação}
\label{subsubsec:progressovacinacao}

Filtrando o conjunto de dados pelas colunas de vacinação, é possível identificar mudanças nos fatores de risco, de modo a perceber os impactos da vacinação no Recife de acordo com seu progresso. Se propõe então observar a situação antes do início da vacinação, assim como depois de um certo progresso da vacinação, para identificar possíveis impactos. 


