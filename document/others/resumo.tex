
% resumo em português
\begin{resumo}[Resumo] 
O combate à COVID-19 se tornou um grande desafio da saúde mundial, tendo mais de 22 milhões de casos e 615 mil mortes por todo o Brasil até a escrita deste trabalho. Em Recife, Pernambuco, até Setembro de 2021, mais de 600 mil casos foram documentados, onde mais de 7 mil resultaram na morte do paciente. Por meio da ampla coleta de dados realizada pela Prefeitura do Recife, este trabalho tem como objetivo comparar a efetividade de diferentes métodos de classificação por aprendizagem de máquina na análise de fatores de risco apresentados pela população, de forma a alertar para chances de casos graves da doença e possível óbito, se baseando nos dados demográficos e sintomáticos dos pacientes de Recife, assim como dados relativos à vacinação em progresso na cidade.
% \noindent %- o resumo deve ter apenas 1 parágrafo e sem recuo de texto na primeira linha, essa tag remove o recuo. Não pode haver quebra de linha.

 \vspace{\onelineskip}
    
 \noindent
 \textbf{Palavras-chaves}: Aprendizagem de máquina. COVID-19. Fatores de risco. Vacinação.
\end{resumo}



% resumo em inglês
\begin{resumo}[Abstract]
\begin{otherlanguage*}{english}

 %\noindent
Combating COVID-19 has become a major global health challenge, with more than 22 million cases and 615,000 deaths throughout Brazil as of the writing of this work. In Recife, Pernambuco, until September 2021, more than 600,000 cases were documented, where more than 7,000 resulted in the patient's death. Through the extensive data collection carried out by the City of Recife, this work aims to compare the effectiveness of different machine learning classification methods in the analysis of risk factors presented by the population, in order to alert to the chances of severe cases of the disease and possible death, based on demographic and symptomatic data of patients in Recife, as well as data on the ongoing vaccination in the city. 



   \vspace{\onelineskip} 
 
   \noindent 
   \textbf{Keywords}: COVID-19. Machine learning. Risk factors. Vaccination. 
 \end{otherlanguage*}
 \end{resumo}
